\documentclass[a4paper]{article}

\usepackage[utf8]{inputenc}
\usepackage[T2A]{fontenc}
\usepackage[english,russian]{babel}

\usepackage{subfig}
\usepackage{graphicx}

\usepackage{float}
% Расширение полей для картинок
\usepackage{chngpage}

% Контейнер для картинок
\newenvironment{figcont}[1]{
    \begin{figure}[H]
        \begin{adjustwidth}{-3cm}{-3cm}
        \centering
        \begin{tabular}{*{#1}c}
}{
        \end{tabular}
        \end{adjustwidth}
    \end{figure}
}

% Картинка в контейнере
\newcommand{\subfig}[2]{
    \subfloat[#1]{\includegraphics[
        width=0.7\textwidth,
        height=6cm,
        keepaspectratio
    ]{figs/#2}}
}

% Без нумерации картинок в контейнере
\captionsetup[subfigure]{labelformat=empty}

\usepackage{array}
% Без выравнивания по ширине
\newcolumntype{N}[1]{>{\raggedright\arraybackslash}p{#1}}

\newenvironment{props}{
    \begin{tabular}{@{}lN{8cm}@{}}
}{
    \end{tabular}
}

% Без отступов первой строки
\usepackage[skip=.8\baselineskip]{parskip}

% Ссылки
\usepackage{hyperref}


\begin{document}

\section*{Проекты}

Репозитории с моими открытыми проектами можно найти здесь:

\begin{tabular}{lN{10cm}}
    GitHub & \url{https://github.com/vpunch}
\end{tabular}

Описание остальных проектов:

\subsection*{CDN}

Проектирование и реализация масштабируемой системы обеспечения защиты и
отказоустойчивости сетевого ресурса. На схеме окружностью обозначен ресурс,
треугольником --- сервер имен, квадратом --- прокси. Прокси выполняет
балансировку нагрузки, кэширование статического контента, скрытие сетевого
адреса ресурса. Сервер имен учитывает географическое расположение клиента и
направляет его к ближайшему прокси. Удаление любого элемента на схеме не
приведет к потере доступа к ресурсу. Установка системы полностью
автоматизирована, она также включает настройку: прав доступа, межсетевого
экрана, SSH-доступа, периодического обновления сертификатов. Результат работы
используется в проектах, демонстрирующих технологию RAIDAtech.

\begin{props}
    Инструменты & Ansible, Nginx, Bind, Let’s Encrypt \\
    Объем, LOC & 720
\end{props}

\begin{figcont}{1}
    \subfig{Схема кластера}{cdn1}
\end{figcont}

\subsection*{Диспетчерская такси}

Кроссплатформенное приложение, позволяющее принимать звонки от клиентов,
отслеживать таксистов, смотреть информацию по машинам и управлять заявками.
Использует REST API для получения данных.

\begin{props}
    Инструменты & Qt, PJSIP, Make, Asterisk, CSS \\
    Объем, LOC & 3720
\end{props}

\begin{figcont}{1}
    \subfig{Интерфейс программы}{taxidisp1}
\end{figcont}

\subsection*{Анализ выживаемости}

Были предоставлены данные мониторинга пациентов, прошедших хирургическое
лечение аортального порока сердца. По этим данным проведен анализ выживаемости.
Синей группе пациентов выполняли протезирование аортального клапана
механическим протезом, оранжевой --- биологическим. Также во время операции
пациенту могли либо не проводить коррекцию митральной регургитации, либо
провести одним из нескольких способов. Была реализована оценка Каплана-Мейера.
Для сравнения кривых, а именно, опровержения однородности распределений, был
реализован критерий log-rank. По результатом анализа лечение с использованием
механического протезирования оказалось наиболее эффективным по критерию
продолжительности жизни. Разницы среди методов коррекции митральной
недостаточности обнаружено не было. Считается, что биологические протезы должны
лучше приживаться, а значит, давать лучшую выживаемость. Обосновать полученные
результаты можно нехваткой опыта биологического протезирования, либо нежеланием
пациентов проводить операцию по замене изношенного протеза.

\begin{props}
    Инструменты & Octave, MariaDB \\
    Объем, LOC & 400 \\
    Код & \url{https://github.com/vpunch/statistics}
\end{props}

\begin{figcont}{2}
    \subfig{Статистика протезирования клапана}{survan1} &
    \subfig{Статистика коррекции регургитации}{survan2}
\end{figcont}

\subsection*{Секундомер}

Создал простой секундомер с динамической индикацией на базе ПЛИС.

\begin{props}
    Инструменты & Quartus, EPM570 \\
    Объем, LOC & 150 \\
    Статус & \closproj
\end{props}

\begin{figcont}{1}
    \subfig{Фотография секундомера}{stopw1}
\end{figcont}

\subsection*{CloudPlusCoin}
% Проект застрял на рефакторинге, библиотеку нельзя использовать

Многопоточная библиотека для работы с CloudCoin API на C++.

\begin{props}
    Инструменты & cURL, Jansson, OpenSSL, SQLite, zlib, CMake \\
    Объем, LOC & 1250
\end{props}

\subsection*{Блог}

Веб-сайт, реализован без фреймворков и имеет REST API, авторизацию, возможность
редактировать публикации, комментировать их, имеется разграничение прав
пользователей.

\begin{props}
    Инструменты & PHP, AJAX, CSS, MariaDB \\
    Объем, LOC & 2900
\end{props}

\end{document}
