\documentclass{article}

\usepackage[utf8]{inputenc}
\usepackage[T2A]{fontenc}
\usepackage[english,russian]{babel}

\usepackage{geometry}
\geometry{a4paper, left=15mm, right=15mm, top=20mm, bottom=15mm}

\usepackage{array}
\usepackage{multirow}
\setlength{\extrarowheight}{1.4mm}
\newcolumntype{N}[1]{>{\raggedright\arraybackslash}p{#1}}

\usepackage{xurl}
\usepackage{graphicx}

\usepackage[skip=.8\baselineskip]{parskip}
\pagenumbering{gobble}


\begin{document}

\begin{minipage}{0.2\textwidth}
    \includegraphics[width=\linewidth]{./my-photo}
\end{minipage}
\hfill
\begin{minipage}{0.78\textwidth}
    \begin{tabular}{N{14cm}}
{\large Панчишин Иван Романович, 24 года}
\\
\texttt{rot1tweiler@gmail.com}
\\
Имею опыт администрирования сетей, работы с машинным обучением, разработки
веб-приложений, чат-ботов и десктоп-приложений. На постоянной основе использую
GNU/Linux. Хочу развиваться в направлении фулстек-разработки веб-приложений:
SQL, C++, Flask, React.
\\
Уровень английского: B2
    \end{tabular}
\end{minipage}

\subsection*{Образование}

\begin{tabular}{lN{5cm}N{10cm}}
    Год & Тип & Место \\\hline
    2005--2016 & Полное общее
    & МКОУ <<Перегребинская СОШ №1>>, Перегребное \\

    2016--2020 & Высшее, бакалавр, с отличием
    & Югорский государственный университет (ЮГУ), Ханты-Мансийск \\

    2020-2021 & Высшее, магистр, первый курс
    & Университет ИТМО, Санкт-Петербург
\end{tabular}

\emph{Подтверждающие документы:}
\url{https://github.com/vpunch/portfolio/blob/main/education/education.pdf}

\subsection*{Карьера}

\begin{tabular}{lN{8cm}N{7cm}}
    Год & Место & Должность \\\hline
    2019--2020 & ООО ИТ-ГРУПП & Сетевой администратор, администратор баз
    данных \\

    2020--2022 & Лаборатория Дискретной оптимизации и формальных методов
    Университета ИТМО & Программист-исследователь \\

    2022--2023 & ООО Научно-технический центр <<Системы управления>> &
    Фулстек-разработчик
\end{tabular}

\subsection*{Проекты}

Примеры проектов, над которыми работал самостоятельно:

\begin{tabular}{N{8cm}N{9cm}}
    TypeScript, React, styled-components, React Query, Lodash, Framer Motion,
    Vite, GraphQL Code Generator, Sentry, Flask, Redis, pyTelegramBotAPI,
    pydantic, Dadata
    & \multirow{1}{=}[-1.5mm]{Сервис для управления торговыми телеграм-ботами,
    интегрированный с ERP-системой на основе Hasura. Включает бэкенд,
    позволяющий при помощи бота совершать заказы, получать рассылку, общаться
    с администратором, и веб-приложение для сборки корзины и оплаты заказа.} \\
    13 000 строк кода & \\
    \url{https://github.com/vpunch/demo/tree/main/tg-store-manager} & \\\hline

    Telegram Bot API, Яндекс Диалоги, VK API, Flask, Yargy, Natasha, NLTK,
    Gensim, scikit-learn, Bash, PostgreSQL, Docker, Let's Encrypt, Apache
    Benchmark, Diagrams.net
    & \multirow{1}{=}[-1.5mm]{Информационный сервис, позволяющий через соц.
    сети, мессенджеры, голосовые помощники или собственное приложение получать
    доступ к расписанию занятий, новостям и справочной информации
    образовательной организации. Проект является дипломным, пояснительную
    записку к нему можно найти по ссылке.} \\
    4500 строк кода & \\
    \url{https://github.com/vpunch/demo/tree/main/UgraSage} & \\\hline

    Qt, CMake, Конфигуратор 1С, Bash, BAT, MSYS2
    & \multirow{1}{=}[-1.5mm]{Кроссплатформенная программа с графическим
    интерфейсом для автоматизации обслуживания баз данных 1C. Проект также
    включает набор скриптов, которые можно вручную добавить в системный
    планировщик (cron или taskschd, в зависимости от ОС). Создавался как
    альтернатива платному и закрытому Обновлятору.} \\
    1700 строк кода & \\
    \url{https://github.com/vpunch/1cmaintenance} & \\ & \\ &
\end{tabular}

\emph{Про другие проекты здесь:}
\url{https://github.com/vpunch/portfolio/blob/main/projects/texout/projects.pdf}

\newpage

\subsection*{Награды}

Основные награды:

\begin{tabular}{lN{7cm}N{7cm}}
    Год & Название &  \\\hline
    2020 & Конгресс молодых ученых & Победитель конкурса докладов для
    поступления в магистратуру \\

    2020 & Чемпионат Югорского государственного университета по стандартам
    WorldSkills & Победитель в компетенции <<Программные решения для
    бизнеса>> \\

    2017, 2018, 2019 & Молодежно научно-практическая конференция
    <<Информационные технологии Югры>> & Победитель и призер
\end{tabular}

\emph{Остальные награды и дипломы:}
\url{https://github.com/vpunch/portfolio/blob/main/rewards/rewards.pdf}

\end{document}
